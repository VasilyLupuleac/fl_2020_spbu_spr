\documentclass[12pt]{article}

\usepackage[utf8]{inputenc}
\usepackage[russian]{babel}
\usepackage{enumerate}
\usepackage{mathdots}
        
\def\to{\rightarrow}
\def\b{\textbf}
\def\t{\textrm}

\newcommand{\divby}{\mathop{\raisebox{-2pt}{$\vdots$}}}

\newenvironment{Problems}{
	\begin{enumerate}[]
	}{       
	\end{enumerate}
}

\begin{document}
	\b{Домашнее задание №5.\ \ \ \ \ \ \ \ \ \ \ \ \ \ \ \ \ \ \ \ \ \ \ \  Лупуляк Василий}
	
	\begin{Problems}
		\item [\fbox{2.}] Построить однозначную КС грамматику, эквивалетную грамматике
		\begin{center}
			$S \to aSbbbb \mid aaaSbb \mid c$\\
		\end{center}
		
		Сделаем так, чтобы первое и второе правила нужно было применять последовательно:
		\begin{center}
			$S \to aSbbbb \mid T \mid c$\\
			$T \to aaaTbb \mid c$
		\end{center}
	  	Покажем, что теперь грамматика однозначна. Пусть слово, принадлежащее грамматике, имеет вид $a^xcb^y$. Пусть его можно получить с помощью $n$ первых правил, $k$ вторых правил и одной замены нетерминала на $c$. Тогда $x = n + 3k, y = 4n + 2k$, откуда 
	  	\begin{center}
	  		$n = \frac{3y - 2x}{10},\ k = \frac{4x - y}{10}$,
	  	\end{center}
  		то есть набор примененных правил определяется однозначно. Все вторые правила применяются после первых, поэтому порядок тоже однозначен, то есть грамматика однозначна.
  		
		\item [\fbox{3.}] Описать язык, порождаемый грамматикой 
		\begin{center}
			$F \to \varepsilon \mid aFaFbF$
		\end{center}
		Этот язык можно описать следующим образом: множество слов из $\{a,b\}^*$, у которых на любом префиксе букв $a$ хотя бы вдвое больше, чем букв $b$, а всего букв $a$ ровно вдвое больше, чем букв $b$. Докажем, что любое слово, порождаемое грамматикой, имеет такой вид. Оба свойства можно доказать по индукции по длине слова (для $\varepsilon$ они выполняются, для $aFaFbF$ второе свойство очевидно, а первое выполняется, так как как перед каждым из трех $F$ количество букв $a$ хотя бы вдвое больше количества букв $b$, а внутри $F$ это верно по предположению индукции).
		
		Теперь докажем, что любое слово из этого языка порождается нашей грамматикой. Будем доказывать индукцией по длине слова, база для пустого слова верна. Переход: рассмотрим первый непустой префикс, на котором количество букв $a$ равно удвоенному количеству букв $b$ (такой точно есть, так как всё слово подходит). Заметим, что он оканчивается на $b$, так как иначе на предыдущем префиксе баланс был бы отрицательным (балансом слова называем разность количества букв $a$ и удвоенного количества букв $b$ в этом слове). Пусть это префикс $wb$ слова $wbv$. Тогда $v$ лежит в языке (так как до него баланс нулевой, поэтому все условия выполняются). Значит, по предположению индукции, слово $v$ порождается грамматикой.
		
		Заметим, что $w$ начинается на $aa$, иначе на втором префиксе баланс был бы отрицательным. Теперь рассмотрим балансы на префиксах $w$ длины хотя бы 2. По выбору $w$, они все положительны. Рассмотрим два случая:
		
		1) Все эти балансы не ниже двух. Тогда если $w = aau$, то для $u$ верно, что баланс на каждом префиксе неотрицателен и баланс на всем $u$ равен нулю. Значит, $u$ порождается грамматикой. Таким образом, исходное слово имеет вид $a\varepsilon aubv$, где $\varepsilon,\ u$ и $v$ порождаются грамматикой, то есть слово порождается грамматикой.
		
		2) На каком-то префиксе баланс равен одному. Тогда рассмотрим последний из таких префиксов. После него обязательно идет буква $a$, так как иначе баланс на следующем префиксе был бы отрицательным. Пусть это префикс $au$ и $w = auas$. Тогда на всех  префиксах от $aua$ до $auas$ баланс не меньше двух, при этом баланс на $aua$ равен двум и баланс на $auas$ равен двум. Значит, $s$ лежит в языке и, следовательно, порождается грамматикой. Аналогичное верно и для $u$. Значит, слово имело вид $auasbv$, где $u,\ s$ и $v$ порождаются грамматикой, поэтому и само слово порождалось грамматикой. Переход доказан.\\
	\end{Problems}
	
\end{document}