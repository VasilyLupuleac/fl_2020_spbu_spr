\documentclass[12pt]{article}
\usepackage[left=2cm,right=2cm,top=2cm,bottom=2cm,bindingoffset=0cm]{geometry}
\usepackage[utf8x]{inputenc}
\usepackage[english,russian]{babel}
\usepackage{cmap}
\usepackage{amssymb}
\usepackage{amsmath}
\usepackage{url}
\usepackage{pifont}
\usepackage{tikz}
\usepackage{verbatim}

\usetikzlibrary{shapes,arrows}
\usetikzlibrary{positioning,automata}
\tikzset{every state/.style={minimum size=0.2cm},
	initial text={}
}

\mathchardef\mhyphen="2D
\newcommand{\divby}{\mathop{\vdots}}
\newcommand{\ndivby}{\mathop{\not\vdots}}

\newenvironment{myauto}[1][3]
{
	\begin{center}
		\begin{tikzpicture}[> = stealth,node distance=#1cm, on grid, very thick]
	}
	{
		\end{tikzpicture}
	\end{center}
}


\begin{document}
	\begin{center} {\LARGE Формальные языки} \end{center}
	
	\begin{center} \Large домашнее задание до 23:59 05.03 \end{center}
	\bigskip
	
	\begin{enumerate}
		\item Доказать или опровергнуть утверждение: произведение двух минимальных автоматов всегда дает минимальный автомат (рассмотреть случаи для пересечения, объединения и разности языков).\\
		
		\textbf{Решение:} для разности это утверждение неверно для равных автоматов из хотя бы двух вершин (так как разность будет приниматься пустым автоматом, а в произведении вершин будет хотя бы две). Для пересечения и объединения рассмотрим следующий контрпример: первый автомат принимает язык $\{a^n \mid n \divby 2\}$, второй принимает $\{a^n \mid n \ndivby 2\}$:
		\begin{myauto}
			\node[state]                        (q_11)                 {$B$};
			\node[state, initial, accepting]    (q_10) [left=of q_11]  {$A$};
			\node[state, initial]               (q_20) [right=of q_11] {$X$};
			\node[state, accepting]             (q_21) [right=of q_20] {$Y$};
			
			\path[->] (q_10)    edge [bend right=15] node  [below] {$a$}       (q_11)
			(q_11)    edge [bend right=15] node  [above] {$a$}       (q_10)
			(q_20)    edge [bend right=15] node  [below] {$a$}       (q_21)
			(q_21)    edge [bend right=15] node  [above] {$a$}       (q_20)
			
			;
		\end{myauto}
		Тогда после удаления недостижимых вершин произведение имеет вид
		\begin{myauto}
			\node[state, initial]     (q_0)                {$AX$};
			\node[state]              (q_1) [right=of q_0] {$BY$};
			
			\path[->] (q_0)    edge [bend right=15] node  [below] {$a$}       (q_1)
			(q_1)              edge [bend right=15] node  [above] {$a$}       (q_0)
			
			;
		\end{myauto}
		или 
		\begin{myauto}
		\node[state, initial, accepting]     (q_0)                {$AX$};
		\node[state, accepting]              (q_1) [right=of q_0] {$BY$};
		
		\path[->] (q_0)    edge [bend right=15] node  [below] {$a$}       (q_1)
		(q_1)              edge [bend right=15] node  [above] {$a$}       (q_0)
		
		;
	\end{myauto}
		(для случаев пересечения и объединения соответственно). В обоих случаях при удалении состояния $BY$ принимаемый язык не меняется, поэтому ни один из автоматов не является минимальным.
		\item Для регулярного выражения:
		\[ (a \mid b)^+ (aa \mid bb \mid abab \mid baba)^* (a \mid b)^+\]
		Построить эквивалентные:
		\begin{enumerate}
			\item Недетерминированный конечный автомат
			\item Недетерминированный конечный автомат без $\varepsilon$-переходов\\
			
			\textbf{Решение:}
			\begin{myauto}
				\node[state]           (q_1)                {$q_1$};
				\node[state,initial]   (q_0) [left=of  q_1] {$q_0$};
				\node[state]           (q_5) [above left=of q_1] {$q_5$};
				\node[state]           (q_6) [above right=of q_5] {$q_6$};
				\node[state]           (q_7) [above right=of q_1] {$q_7$};
				\node[state]           (q_2) [below left=of q_1] {$q_2$};
				\node[state]           (q_4) [below right=of q_1] {$q_4$};
				\node[state]           (q_3) [below left=of q_4] {$q_3$};
				\node[state,accepting] (q_8) [right=of q_1] {$q_8$};
				
				\path[->] (q_0) edge [loop above] node [above] {$a, b$} ()
				edge              node [above] {$a, b$}       (q_1)
				(q_1) edge        [bend right=15] node  [left] {$a$}       (q_2)
				edge        [bend right=15] node  [right] {$b$}       (q_5)
				edge              node [above] {$a,b$}       (q_8)
				(q_2) edge        node [left] {$b$}       (q_3)
				edge [bend right=15] node [right] {$a$} (q_1)
				(q_3) edge              node [right] {$a$}       (q_4)
				(q_4) edge              node [right] {$b$}       (q_1)
				(q_5) edge        node [left] {$a$}       (q_6)
				edge [bend right=15] node [left] {$b$} (q_1)
				(q_6) edge              node [right] {$b$}       (q_7)
				(q_7) edge              node [right] {$a$}       (q_1)
				(q_8) edge [loop above] node [above] {$a, b$} ()
				;
			\end{myauto}
			\item Минимальный полный детерминированный конечный автомат\\
			
			\textbf{Решение:} заметим, что все подходяющие под выражение слова имеют длину не менее двух символов. С другой стороны, все такие слова подходят под выражение (т.к. имеют вид $(a\mid b)(a \mid b)^*(a \mid b)$, то есть подходят под $(a\mid b)^+\varepsilon(a \mid b)^+$). Значит, минимальный автомат выглядит следующим образом:
			\begin{myauto}
				\node[state, accepting]    (q_2)               {$q_2$};
				\node[state]               (q_1) [left=of q_2] {$q_1$};
				\node[state, initial]      (q_0) [left=of q_1] {$q_0$};
				
				\path[->] (q_0)    edge node [above] {$a, b$}  (q_1)
				(q_1)              edge node [above] {$a, b$}  (q_2)
				(q_2) edge [loop above] node [above] {$a, b$}    ()
				
				;
			\end{myauto}
		\end{enumerate}
		\item Построить регулярное выражение, распознающее тот же язык, что и автомат:
		\begin{myauto}
			\node[state]           (q_2)                {$q_2$};
			\node[state,initial]   (q_0) [left=of  q_2] {$q_0$};
			\node[state]           (q_1) [above=of q_2] {$q_1$};
			\node[state]           (q_3) [below=of q_2] {$q_3$};
			\node[state,accepting] (q_4) [right=of q_2] {$q_4$};
			
			\path[->] (q_0) edge [loop above] node [above] {$a, b, c$} ()
			edge              node [above] {$a$}       (q_1)
			edge              node [above] {$b$}       (q_2)
			edge              node [above] {$c$}       (q_3)
			(q_1) edge [loop above] node [above] {$b, c$}    ()
			edge              node [above] {$a$}       (q_4)
			(q_2) edge [loop above] node [above] {$a, c$}    ()
			edge              node [above] {$b$}       (q_4)
			(q_3) edge [loop above] node [above] {$a, b$}    ()
			edge              node [above] {$c$}       (q_4)
			;
		\end{myauto}
		\textbf{Решение:} если автомат распознает некоторое слово $\omega x,\ \omega \in \{ a, b, c \}^*,\ x \in \{a, b, c\}$, то $|\omega|_x \geqslant 1$, так как выход из $q_0$ должен был быть совершен по символу $x$. Несложно заметить, что любое слово $\omega x$ с таким свойством принимается автоматом (выход из $q_0$ нужно произвести по последнему вхождению $x$ в $\omega$). Это свойство можно записать в виде регулярного выражения следующим образом:
		\[ [a \mhyphen c]^*(a[a \mhyphen c]^*a \mid b[a \mhyphen c]^*b \mid c[a \mhyphen c]^*c)\]
		\item Определить, является ли автоматным язык $\{ \omega \omega^r \mid \omega \in \{ 0, 1 \}^* \}$. Если является --- построить автомат, иначе --- доказать.
		
		\textbf{Решение:} предположим, что язык является автоматным. Тогда, по теореме о накачке, для достаточно большого $N$ $\exists x,y,z: 1^N001^N = xyz,\ |y| > 0,\ |xy| < N$ и $xyyz$ тоже лежит в языке. Но $xyyz = 1^{(N + |y|)}001^N$, то есть это слово не является палиндромом, что приводит к противоречию. Таким образом, язык не является автоматным.
		
		\item Определить, является ли автоматным язык $\{ u a a v \mid u, v \in \{ a, b \}^* , |u|_b \geq |v|_a \}$. Если является --- построить автомат, иначе --- доказать.\\
		
		\textbf{Решение:} предположим, что язык является автоматным. Применим теорему о накачке к слову $b^naa(ba)^n = xyz$ (для $n$ из условия теоремы). Тогда $y \in\{b\}^*$, так как $|xy| \leqslant n$. Тогда слово $xz = b^{(n - |y|)}aa(ba)^n$ должно лежать в языке, но оно не лежит, так как это слово единственным способом представляется в виде  $uaav$ и при этом $|u|_b = n - |y| < n = |v|_a$. Получаем противоречие, тем самым доказав, что язык не является автоматным.
	\end{enumerate}
	
	\newpage
	
	\begin{center}
		\Large{Пример применения алгоритма минимизации}
	\end{center}
	
	\bigskip
	
	Минимизируем данный автомат:
	
	\begin{center}
		\begin{tikzpicture}[> = stealth,node distance=3cm, on grid]
		\node[state]           (q_2)                      {C};
		\node[state,initial]   (q_0) [above left=of q_2]  {A};
		\node[state]           (q_1) [below left=of q_2]  {B};
		\node[state]           (q_3) [right=of q_2]       {D};
		\node[state]           (q_4) [above right=of q_3] {E};
		\node[state,accepting] (q_5) [below right=of q_3] {F};
		\node[state,accepting] (q_6) [above right=of q_5] {G};
		
		\path[->] (q_0) edge [bend left=15]  node [right] {$1$} (q_1)
		edge                 node [above] {$0$} (q_2)
		(q_1) edge [bend left=15]  node [left]  {$1$} (q_0)
		edge                 node [below] {$0$} (q_2)
		(q_2) edge [bend right=15] node [below] {$1$} (q_3)
		edge [bend left=15]  node [above] {$0$} (q_3)
		(q_3) edge                 node [below] {$1$} (q_5)
		edge                 node [above] {$0$} (q_4)
		(q_4) edge                 node [above] {$1$} (q_6)
		edge                 node [right] {$0$} (q_5)
		(q_5) edge [loop below]    node         {$1$} ()
		edge [loop left]     node         {$0$} ()
		(q_6) edge                 node [below] {$1$} (q_5)
		edge [loop right]    node         {$0$} ();
		\end{tikzpicture}
	\end{center}
	
	Автомат полный, в нем нет недостижимых вершин --- продолжаем.
	
	Строим обратное $\delta$ отображение.
	
	\begin{tabular}{c|c|c}
		$\delta^{-1}$ & 0 & 1 \\ \hline
		A & --- & B \\
		B & --- & A \\
		C & A B & --- \\
		D & C & C \\
		E & D & --- \\
		F & E F & D F G \\
		G & G & E
	\end{tabular}
	
	Отмечаем в таблице и добавляем в очередь пары состояний, различаемых словом $\varepsilon$: все пары, один элемент которых --- терминальное состояние, а второй --- не терминальное состояние. Для данного автомата это пары
	
	$(A, F), (B, F), (C, F), (D, F), (E,F), (A, G), (B, G), (C, G), (D, G), (E, G)$
	
	Дальше итерируем процесс определения неэквивалентных состояний, пока очередь не оказывается пуста.
	
	$(A, F)$ не дает нам новых неэквивалентных пар. Для $(B, F)$ находится 2 пары: $(A, D), (A, G)$. Первая пара не отмечена в таблице --- отмечаем и добавляем в очередь. Вторая пара уже отмечена в таблице, значит, ничего делать не надо. Переходим к следующей паре из очереди. Итерируем дальше, пока очередь не опустошится.
	
	Результирующая таблица (заполнен только треугольник, потому что остальное симметрично) и порядок добавления пар в очередь.
	
	\begin{tabular}{c|cc|cc|cc|c}
		& A & B & C & D & E & F & G \\ \hline
		A &&&&&&& \\
		B &&&&&&& \\ \hline
		C & \checkmark & \checkmark &&&&& \\
		D & \checkmark & \checkmark & \checkmark &&&& \\ \hline
		E & \checkmark & \checkmark & \checkmark & \checkmark &&& \\
		F & \checkmark & \checkmark & \checkmark & \checkmark & \checkmark && \\ \hline
		G & \checkmark & \checkmark & \checkmark & \checkmark & \checkmark && \\
	\end{tabular}
	
	Очередь:
	
	$
	(A, F), (B, F), (C, F), (D, F), (E,F), (A, G), (B, G), (C, G), (D, G), (E, G),
	$
	
	$
	(B, D), (A, D), (A, E), (B, E), (C, E), (C, D), (D, E), (A,C), (B, C))
	$
	
	В таблице выделились классы эквивалентных вершин: $\{A, B\}, \{C\}, \{D\}, \{E\}, \{F,G\}$. Остается только нарисовать результирующий автомат с вершинами-классами. Переходы добавляются тогда, когда из какого-нибудь состояния первого класса есть переход в какое-нибудь состояние второго класса. Минимизированный автомат:
	
	\begin{center}
		\begin{tikzpicture}[> = stealth,node distance=3cm, on grid]
		\node[state,initial]   (q_01)                     {AB};
		\node[state]           (q_2)  [right=of q_01]      {C};
		\node[state]           (q_3)  [right=of q_2]       {D};
		\node[state]           (q_4)  [above right=of q_3] {E};
		\node[state,accepting] (q_56) [below right=of q_3] {FG};
		
		\path[->] (q_01) edge [loop above]    node [above] {$1$} ()
		edge                 node [above] {$0$} (q_2)
		(q_2)  edge [bend right=15] node [below] {$1$} (q_3)
		edge [bend left=15]  node [above] {$0$} (q_3)
		(q_3)  edge                 node [below] {$1$} (q_56)
		edge                 node [above] {$0$} (q_4)
		(q_4)  edge [bend right=15] node [left]  {$1$} (q_56)
		edge [bend left=15]  node [right] {$0$} (q_56)
		(q_56) edge [loop below]    node         {$1$} ()
		edge [loop left]     node         {$0$} ();
		\end{tikzpicture}
	\end{center}
	
\end{document}