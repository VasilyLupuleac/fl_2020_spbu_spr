\documentclass[12pt]{article}

\usepackage[utf8]{inputenc}
\usepackage[russian]{babel}
\usepackage{enumerate}
\usepackage{mathdots}
        
\def\to{\rightarrow}
\def\b{\textbf}
\def\t{\textrm}

\newcommand{\divby}{\mathop{\raisebox{-2pt}{$\vdots$}}}

\newenvironment{Problems}{
	\begin{enumerate}[]
	}{       
	\end{enumerate}
}

\begin{document}
	\b{Домашнее задание №5.\ \ \ \ \ \ \ \ \ \ \ \ \ \ \ \ \ \ \ \ \ \ \ \  Лупуляк Василий}
	
	\begin{Problems}
		\item [\fbox{2.}] Привести в НФХ грамматику 
		\begin{center}
			$S$ -- стартовый нетерминал\\
			$S \to RS \mid R$\\
			$R \to aSb \mid cRd \mid ab \mid cd \mid \varepsilon$
		\end{center}
		Добавим новый стартовый нетерминал: 
		\begin{center}
			$S'$ -- стартовый нетерминал\\
			$S' \to S$\\
			$S \to RS \mid R$\\
			$R \to aSb \mid cRd \mid ab \mid cd \mid \varepsilon$
		\end{center}
		Удалим длинные правила: 
		\begin{center}
			$S' \to S$\\
			$S \to RS \mid R$\\
			$R \to aS_b \mid cR_d \mid ab \mid cd \mid \varepsilon$\\
			$R_d \to Rd$\\
			$S_b \to Sb$
		\end{center}
		Удалим $\varepsilon$-правила:
		\begin{center}
			$S' \to S \mid \varepsilon$\\
			$S \to RS \mid R \mid S$\\
			$R \to aS_b \mid cR_d \mid ab \mid cd$\\
			$R_d \to Rd \mid d$\\
			$S_b \to Sb \mid b$
		\end{center}
		Удалим цепные правила:
		\begin{center}
			$S' \to RS \mid aS_b \mid cR_d \mid ab \mid cd \mid \varepsilon$\\
			$S \to RS \mid aS_b \mid cR_d \mid ab \mid cd$\\
			$R \to aS_b \mid cR_d \mid ab \mid cd$\\
			$R_d \to Rd \mid d$\\
			$S_b \to Sb \mid b$
		\end{center}
		Бесполезных нетерминалов нет. Осталось привести правые части к нужному виду:
		\begin{center}
			$S'$ -- стартовый нетерминал\\
			$S' \to RS \mid AS_b \mid CR_d \mid AB \mid CD \mid \varepsilon$\\
			$S \to RS \mid AS_b \mid CR_d \mid AB \mid CD$\\
			$R \to AS_b \mid CR_d \mid AB \mid CD$\\
			$R_d \to RD \mid d$\\
			$S_b \to SB \mid b$\\
			$A \to a$\\
			$B \to b$\\
			$C \to c$\\
			$D \to d$
		\end{center}
		\item [\fbox{3.}] Построить КС грамматику языка $\{a^m b^n \mid m + n > 0,\ (m + n) \divby 2\}$.\\
		Слова имеют следующий вид: либо каждой из букв нечетное количество, либо каждой четное и одной из букв хотя бы две. Значит, любое слово получается из слов $ab,\ aa,\t{или}\ bb$ добавлением четного количества букв $a$ слева и букв $b$ справа. Значит, грамматику можно записать следующим образом: 
		\begin{center}
			$S$ -- стартовый нетерминал\\
			$S \to aS'b \mid aaS' \mid S'bb$\\
			$S' \to aaS' \mid S'bb \mid \varepsilon$\\
		\end{center}		
	\end{Problems}
	
\end{document}